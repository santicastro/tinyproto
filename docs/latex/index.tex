This is Tiny protocol implementation for microcontrollers (Arduino, Stellaris).\hypertarget{index_introduction}{}\section{Introduction}\label{index_introduction}
This protocol is intended to be used in low-\/memory systems, like microcontrollers (Stellaris, Arduino). It is also can be compiled for desktop Linux systems, and it is possible to build it for Windows.\hypertarget{index_api}{}\section{Tiny Protocol A\+PI}\label{index_api}
Library A\+PI supports C-\/\+Style functions -\/ the basic A\+PI, and C++ A\+PI, which provides high level easy to use classes. Please refer to documentation. Basically Tiny\+Proto library provides 4 different protocol implementations\+:

\hyperlink{group__HDLC__API}{Tiny H\+D\+LC protocol A\+PI functions} This is basis, which all other protocols implementations are built on top of. Hdlc api provides basic input/output operations, like F\+CS, escape sequencies, etc. There is no C++ wrappers for it.

\hyperlink{group__LIGHT__API}{Tiny light protocol A\+PI functions} This is simple protocol, based on H\+D\+LC api. It simplifies sending and receiving messages on small controllers, and at the same time it has low memory consumption (800 bytes of flash). But, be careful since light version doesn\textquotesingle{}t have any confirmation from remote side.

\hyperlink{group__HALF__DUPLEX__API}{Tiny Half Duplex A\+PI functions} This is simple implementation with A\+CK from remote side. It needs a little more memory than light version, but it is good still for small microcontrollers. The disadvantages of hd protocol is\+: it doesn\textquotesingle{}t use official H\+D\+LC frame format, each frame being sent must be confirmed before sending next frame. This can cause slow down in communication.

\hyperlink{group__FULL__DUPLEX__API}{Tiny Full Duplex A\+PI functions} This is the heaviest protocol implementation in the library. For atmega controllers it requires 7\+KiB of flash, and at least 700 bytes to operation with 64-\/byte length frames. Unlike hd version, fd version of protocol is faster, as it doesn\textquotesingle{}t wait for confirmation before sending next frame (thanks to window, specified in H\+D\+LC specs). Fd version of protocol uses official H\+D\+LC frame format, and implements U-\/frames (S\+A\+BM,UA), S-\/frames (RR,R\+EJ), I-\/frames.\hypertarget{index_packet}{}\section{Packet Structure}\label{index_packet}
H\+D\+LC frame format\+: 
\begin{DoxyPre}
     8        any len    None/8/16/32     8
 |   7E   |    DATA     |    FCS     |   7E   |
\end{DoxyPre}


Full duplex hdlc uses standard hdlc frame format\+: 
\begin{DoxyPre}
     8     ADDR  CONTROL     any len    None/8/16/32     8
 |   7E   | FF | hdlc ctl | USER DATA  |    FCS     |   7E   |
\end{DoxyPre}



\begin{DoxyItemize}
\item 7E is packet delimiter (commonly used on layer 2) as defined in H\+D\+L\+C/\+P\+PP framing. packet delimiter is used by the protocol to find first and last byte of the frame being transmitted.
\item U\+S\+ER D\+A\+TA of any length. This field contains user data with replaced 0x7D and 0x7E bytes by special sequenced as defined in H\+D\+L\+C/\+P\+PP framing. Length of data is now limited on buffer used to receive frames and/or 32767 bytes (Tiny Protocol using 16-\/bit field to store frame length).
\item F\+CS is standard checksum. Depending on your selection, this is 8-\/bit, 16-\/bit or 32-\/bit field, or it can be absent at all. Refer to R\+F\+C1662 for examples. F\+CS field is also optional and can be disabled. But in this case transport errors are not detected.
\end{DoxyItemize}\hypertarget{index_callback}{}\section{User-\/defined callbacks}\label{index_callback}
\hyperlink{group__HDLC__API}{Tiny H\+D\+LC protocol A\+PI functions} needs 3 callback functions, defined by a user (you may use any function names you need).

H\+D\+LC callbacks\+: 
\begin{DoxyCode}
\textcolor{keywordtype}{int} write\_func\_cb(\textcolor{keywordtype}{void} *user\_data, \textcolor{keyword}{const} \textcolor{keywordtype}{void} *data, \textcolor{keywordtype}{int} len);
\textcolor{keywordtype}{int} on\_frame\_read(\textcolor{keywordtype}{void} *user\_data, \textcolor{keywordtype}{void} *data, \textcolor{keywordtype}{int} len);
\textcolor{keywordtype}{int} on\_frame\_sent(\textcolor{keywordtype}{void} *user\_data, \textcolor{keyword}{const} \textcolor{keywordtype}{void} *data, \textcolor{keywordtype}{int} len);
\end{DoxyCode}



\begin{DoxyItemize}
\item write\+\_\+func\+\_\+cb() is called by H\+D\+LC implementation every time, it needs to send bytes to TX channel
\item on\+\_\+frame\+\_\+read() is called by H\+D\+LC implementation every time, new frame arrives and checksum is correct.
\item on\+\_\+frame\+\_\+sent() is called by H\+D\+LC implementation every time, new frame is sent to TX. H\+D\+LC protocol requires only write\+\_\+func\+\_\+cb() to be defined. Other callbacks are optional. As for RX processes, your application code is responsible for reading data from RX line, then all you need to do, is to pass received bytes to H\+D\+LC implementation for processing via \hyperlink{group__HDLC__API_ga911a3f1cb32dd6cadd00223e0097642c}{hdlc\+\_\+run\+\_\+rx()}.
\end{DoxyItemize}

All higher level protocols (\hyperlink{group__LIGHT__API}{Tiny light protocol A\+PI functions}, \hyperlink{group__HALF__DUPLEX__API}{Tiny Half Duplex A\+PI functions}, \hyperlink{group__FULL__DUPLEX__API}{Tiny Full Duplex A\+PI functions}) needs 4 callback functions, defined by a user\+: read\+\_\+func\+\_\+cb() is added. The list of callbacks\+:


\begin{DoxyCode}
\textcolor{keywordtype}{int} write\_func\_cb(\textcolor{keywordtype}{void} *user\_data, \textcolor{keyword}{const} \textcolor{keywordtype}{void} *data, \textcolor{keywordtype}{int} len);
\textcolor{keywordtype}{int} read\_func\_cb(\textcolor{keywordtype}{void} *user\_data, \textcolor{keywordtype}{void} *data, \textcolor{keywordtype}{int} len);
\textcolor{keywordtype}{int} on\_frame\_read(\textcolor{keywordtype}{void} *user\_data, \textcolor{keywordtype}{void} *data, \textcolor{keywordtype}{int} len);
\textcolor{keywordtype}{int} on\_frame\_sent(\textcolor{keywordtype}{void} *user\_data, \textcolor{keyword}{const} \textcolor{keywordtype}{void} *data, \textcolor{keywordtype}{int} len);
\end{DoxyCode}


Unlike H\+D\+LC implementation, higher level protocols use different approach. They control both TX and RX channels, for example, to transparently send A\+CK frames, etc. That\textquotesingle{}s why higher level protocols need to read\+\_\+func\+\_\+cb to be defined\+:


\begin{DoxyItemize}
\item read\+\_\+func\+\_\+cb() is called by higher level protocol implementation every time, it needs to read bytes from RX channel.
\end{DoxyItemize}\hypertarget{index_arduino_section}{}\section{Arduino related documentation}\label{index_arduino_section}
\hyperlink{arduino}{Arduino related A\+PI (C++)} 